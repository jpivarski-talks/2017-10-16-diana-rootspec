\documentclass[aspectratio=169]{beamer}

\mode<presentation>
{
  \usetheme{default}
  \usecolortheme{default}
  \usefonttheme{default}
  \setbeamertemplate{navigation symbols}{}
  \setbeamertemplate{caption}[numbered]
  \setbeamertemplate{footline}[frame number]  % or "page number"
  \setbeamercolor{frametitle}{fg=white}
  \setbeamercolor{footline}{fg=black}
} 

\usepackage[english]{babel}
\usepackage[utf8x]{inputenc}
\usepackage{tikz}
\usepackage{courier}
\usepackage{array}
\usepackage{bold-extra}
\usepackage{minted}
\usepackage[thicklines]{cancel}

\xdefinecolor{dianablue}{rgb}{0.18,0.24,0.31}
\xdefinecolor{darkblue}{rgb}{0.1,0.1,0.7}
\xdefinecolor{darkgreen}{rgb}{0,0.5,0}
\xdefinecolor{darkgrey}{rgb}{0.35,0.35,0.35}
\xdefinecolor{darkorange}{rgb}{0.8,0.5,0}
\xdefinecolor{darkred}{rgb}{0.7,0,0}
\definecolor{darkgreen}{rgb}{0,0.6,0}
\definecolor{mauve}{rgb}{0.58,0,0.82}

\title[2017-10-16-diana-rootspec]{Describing the ROOT format with a DSL}
\author{Jim Pivarski}
\institute{Princeton University -- DIANA}
\date{October 16, 2017}

\begin{document}

\logo{\pgfputat{\pgfxy(0.11, 7.4)}{\pgfbox[right,base]{\tikz{\filldraw[fill=dianablue, draw=none] (0 cm, 0 cm) rectangle (50 cm, 1 cm);}\includegraphics[height=1 cm]{diana-hep-logo.png}}}}

\begin{frame}
  \titlepage
\end{frame}

% Uncomment these lines for an automatically generated outline.
%\begin{frame}{Outline}
%  \tableofcontents
%\end{frame}

%%%%%%%%%%%%%%%%%%%%%%%%%%%%%%%%%%%%%%%%%%%%%%%%%%%%%%%

%%%% START

\begin{frame}{}
\vspace{0.5 cm}
\begin{center}
\large ROOT is a file format.
\end{center}

\begin{itemize}
\item It's like HDF5 in that it organizes data objects in a filesystem-like structure.
\item It's like Avro in that it defines the structure of the classes it stores.
\item It's like Parquet in that it can split classes into columns for efficient access.
\item It's more like Arrow/Feather in the way that it implements splitting.
\item It's like Pickle in that its data model encompasses an entire language (C++ rather than Python).
\item It's like FITS in that it was developed by a scientific community for that community.
\item It's unlike the above in that it has only one implementation (until recently).
\end{itemize}
\end{frame}

\begin{frame}{Number of implementations}
\vspace{0.25 cm}
\begin{center}
\renewcommand{\arraystretch}{1.5}
\begin{tabular}{l c c c}
& inception & specification & implementations \\\hline
Protocol buffers &  & \href{https://developers.google.com/protocol-buffers/docs/encoding}{\textcolor{blue}{\tiny https://developers.google.com/protocol-buffers/docs/encoding}} & 20 \\
Thrift &  & $\surd$ & 15 \\
Avro &  & $\surd$ & 13 \\
Parquet & 2013 & \href{http://parquet.apache.org/documentation/latest/}{\textcolor{blue}{\tiny http://parquet.apache.org/documentation/latest/}} & 5 \\
Arrow/Feather & 2016 & \href{https://arrow.apache.org/docs/memory_layout.html}{\textcolor{blue}{\tiny https://arrow.apache.org/docs/memory\_layout.html}} & 7 \\
netCDF,HDF4/5 & 1992 & \href{https://support.hdfgroup.org/HDF5/doc/H5.format.html}{\textcolor{blue}{\tiny https://support.hdfgroup.org/HDF5/doc/H5.format.html}} & 35 \\
FITS & 1981 & \href{https://fits.gsfc.nasa.gov/standard30/fits_standard30aa.pdf}{\textcolor{blue}{\tiny https://fits.gsfc.nasa.gov/standard30/fits\_standard30aa.pdf}} & 38 \\
Pickle & 1996 & no? & 3? \\
ROOT & 1995 & & 5
\end{tabular}
\end{center}
\end{frame}

\href{}{\textcolor{blue}{\tiny }}

\end{document}
