\documentclass[aspectratio=169]{beamer}

\mode<presentation>
{
  \usetheme{default}
  \usecolortheme{default}
  \usefonttheme{default}
  \setbeamertemplate{navigation symbols}{}
  \setbeamertemplate{caption}[numbered]
  \setbeamertemplate{footline}[frame number]  % or "page number"
  \setbeamercolor{frametitle}{fg=white}
  \setbeamercolor{footline}{fg=black}
} 

\usepackage[english]{babel}
\usepackage[utf8x]{inputenc}
\usepackage{tikz}
\usepackage{courier}
\usepackage{array}
\usepackage{bold-extra}
\usepackage{minted}
\usepackage[thicklines]{cancel}
\usepackage{fancyvrb}

\xdefinecolor{dianablue}{rgb}{0.18,0.24,0.31}
\xdefinecolor{darkblue}{rgb}{0.1,0.1,0.7}
\xdefinecolor{darkgreen}{rgb}{0,0.3,0}
\xdefinecolor{darkgrey}{rgb}{0.35,0.35,0.35}
\xdefinecolor{darkorange}{rgb}{0.8,0.5,0}
\xdefinecolor{brown}{rgb}{0.6,0.3,0}
\definecolor{darkgreen}{rgb}{0,0.6,0}
\definecolor{mauve}{rgb}{0.58,0,0.82}

\title[2017-10-16-diana-rootspec]{Describing the ROOT format with a DSL}
\author{Jim Pivarski}
\institute{Princeton University -- DIANA}
\date{October 16, 2017}

\begin{document}

\logo{\pgfputat{\pgfxy(0.11, 7.4)}{\pgfbox[right,base]{\tikz{\filldraw[fill=dianablue, draw=none] (0 cm, 0 cm) rectangle (50 cm, 1 cm);}\includegraphics[height=1 cm]{diana-hep-logo.png}}}}

\begin{frame}
  \titlepage
\end{frame}

% Uncomment these lines for an automatically generated outline.
%\begin{frame}{Outline}
%  \tableofcontents
%\end{frame}

%%%%%%%%%%%%%%%%%%%%%%%%%%%%%%%%%%%%%%%%%%%%%%%%%%%%%%%

%%%% START

\begin{frame}{}
\vspace{0.75 cm}
\begin{center}
\large ROOT is a file format.
\end{center}

\begin{itemize}\setlength{\itemsep}{0.2 cm}
\item It's like HDF5 in that it organizes data objects in a filesystem-like structure.
\item It's like Avro in that it defines the structure of the classes it stores.
\item It's like Parquet in that it can split classes into columns for efficient access.
\item Although it's more like Arrow/Feather in the way that it implements splitting.
\item It's like Pickle in that its data model encompasses an entire language (C++ rather than Python).
\item It's like FITS in that it was developed by a scientific community for that community.
\item It's unlike most of the above in that it doesn't have a formal specification.
\end{itemize}
\end{frame}

\begin{frame}{Specifications and implementations: what I could find}
\vspace{0.5 cm}
\begin{columns}
\column{1.1\linewidth}
\renewcommand{\arraystretch}{1.5}
\begin{tabular}{l c c c}
& inception & specification & implementations \\\hline

FITS & 1981 & \href{https://fits.gsfc.nasa.gov/standard30/fits_standard30aa.pdf}{\textcolor{blue}{\tiny https://fits.gsfc.nasa.gov/standard30/fits\_standard30aa.pdf}} & 38 \\

netCDF,HDF4/5 & 1992 & \href{https://support.hdfgroup.org/HDF5/doc/H5.format.html}{\textcolor{blue}{\tiny https://support.hdfgroup.org/HDF5/doc/H5.format.html}} & 35 \\

ROOT & 1995 & {\tiny some class headers like \href{https://root.cern.ch/doc/master/classTFile.html}{\textcolor{blue}{TFile}} and \href{https://root.cern.ch/doc/master/classTKey.html}{\textcolor{blue}{TKey}}; not enough info to read a file} & 6 \\

Pickle & 1996 & {\tiny {\it implementation} changes: 1$\to$2 \href{http://legacy.python.org/dev/peps/pep-0307/}{\textcolor{blue}{PEP-307}}, 3$\to$4 \href{https://www.python.org/dev/peps/pep-3154/}{\textcolor{blue}{PEP-3154}}; not a real spec} & 4 \\

Protocol buffers & 2001 & \href{https://developers.google.com/protocol-buffers/docs/encoding}{\textcolor{blue}{\tiny https://developers.google.com/protocol-buffers/docs/encoding}} & 20 \\

Thrift & 2007 & {\tiny {\bf UNOFFICIAL:} \href{https://erikvanoosten.github.io/thrift-missing-specification/}{\textcolor{blue}{\tiny https://erikvanoosten.github.io/thrift-missing-specification/}}} & 15 \\

Avro & 2009 & \href{http://avro.apache.org/docs/current/spec.html}{\textcolor{blue}{\tiny http://avro.apache.org/docs/current/spec.html}} & 13 \\

Parquet & 2013 & \href{http://parquet.apache.org/documentation/latest/}{\textcolor{blue}{\tiny http://parquet.apache.org/documentation/latest/}} & 5 \\

Arrow/Feather & 2016 & \href{https://arrow.apache.org/docs/memory_layout.html}{\textcolor{blue}{\tiny https://arrow.apache.org/docs/memory\_layout.html}} & 7 \\
\end{tabular}
\end{columns}
\end{frame}

\begin{frame}{}
\vspace{1 cm}
\begin{columns}[t]
\column{0.5\linewidth}
\mbox{ } \hfill \underline{\large Why not specify?} \hfill \mbox{ }

\vspace{0.25 cm}
\begin{itemize}
\item inhibits development
\item human-readable documents get out of date
\item personnel already limited
\item streamer info already specifies most classes dynamically
\item ROOT C++ implementation must be primary
\end{itemize}

\column{0.5\linewidth}
\mbox{ } \hfill \underline{\large Why specify?} \hfill \mbox{ }

\vspace{0.25 cm}
\begin{itemize}
\item better data preservation
\item clarifies invariants that are hard to express or not local in code
\item formal process for adding I/O features
\item allows alternate I/O projects to maintain themselves
\item may be descriptive, rather than prescriptive
\end{itemize}

\end{columns}
\end{frame}

\begin{frame}{What alternate implementations?}
\begin{center}
\renewcommand{\arraystretch}{1.5}
\begin{tabular}{l c p{4.5 cm} l}
project & language & purpose & maintainer \\\hline
ROOT & C++ & main project & the ROOT Team (Philippe Canal) \\
JsRoot & Javascript & interacting with ROOT in the browser or standalone & the ROOT Team (Sergey Linev) \\
RIO & C++ & embedded in GEANT-4 & Guy Barrand? \\
root4j & Java & Spark and other Big Data & Viktor Khristenko \\
rootio & Go & go-hep ecosystem in Go & Sebastien Binet \\
uproot & Python & BulkIO-style Numpy access, pip-installable root\_numpy, understanding ROOT I/O, prototyping & Jim Pivarski (me)
\end{tabular}
\end{center}
\end{frame}

\begin{frame}[fragile]{Sample code from uproot}
\vspace{0.5 cm}
\small
\begin{minted}{python}
fNbytes, fVersion, fObjlen, fDatime, fKeylen, fCycle = \
                                         file.readfields("!ihiIhh")

if fVersion > 1000:
    fSeekKey, fSeekPdir = file.readfields("!qq")   # 64-bit
else:
    fSeekKey, fSeekPdir = file.readfields("!ii")   # 32-bit

fClassName = file.readstring()  # byte or int32 size prefix
fName      = file.readstring()
fTitle     = file.readstring()
\end{minted}

\vspace{0.25 cm}
\begin{center}
\uncover<2->{\textcolor{darkblue}{\Large This is imperative Python code, but it doesn't need to be.}}
\end{center}
\end{frame}

\begin{frame}[fragile]{As a declarative specification}
\vspace{0.4 cm}

\begin{columns}
\column{0.45\linewidth}
\vspace{-0.25 cm}
\scriptsize
\begin{onlyenv}<1>
\begin{Verbatim}[commandchars=\\\{\}]
\textcolor{darkgreen}{\textbf{TKey:}}
  \textcolor{darkgreen}{\textbf{assert:}}
    \textcolor{blue}{\textbf{-}} \textcolor{brown}{$size == fKeylen}  \textcolor{mauve}{# check!}
  \textcolor{darkgreen}{\textbf{members:}}
    \textcolor{blue}{\textbf{-}} \textcolor{darkgreen}{\textbf{fNbytes:}} \textcolor{brown}{int32}
    \textcolor{blue}{\textbf{-}} \textcolor{darkgreen}{\textbf{fVersion:}} \textcolor{brown}{int16}
    \textcolor{blue}{\textbf{-}} \textcolor{darkgreen}{\textbf{fObjlen:}} \textcolor{brown}{int32}
    \textcolor{blue}{\textbf{-}} \textcolor{darkgreen}{\textbf{fDatime:}} \textcolor{brown}{int32}
    \textcolor{blue}{\textbf{-}} \textcolor{darkgreen}{\textbf{fKeylen:}} \textcolor{brown}{int16}
    \textcolor{blue}{\textbf{-}} \textcolor{darkgreen}{\textbf{fCycle:}} \textcolor{brown}{int16}
    \textcolor{blue}{\textbf{-}} \textcolor{darkgreen}{\textbf{if:}}
        \textcolor{blue}{\textbf{-}} \textcolor{darkgreen}{\textbf{case:}} \textcolor{brown}{version > 1000}
          \textcolor{darkgreen}{\textbf{then:}}
            \textcolor{blue}{\textbf{-}} \textcolor{darkgreen}{\textbf{fSeekKey:}} \textcolor{brown}{int64}  \textcolor{mauve}{# big}
            \textcolor{blue}{\textbf{-}} \textcolor{darkgreen}{\textbf{fSeekPdir:}} \textcolor{brown}{int64}
        \textcolor{blue}{\textbf{-}} \textcolor{darkgreen}{\textbf{else:}}
            \textcolor{blue}{\textbf{-}} \textcolor{darkgreen}{\textbf{fSeekKey:}} \textcolor{brown}{int32}  \textcolor{mauve}{# small}
            \textcolor{blue}{\textbf{-}} \textcolor{darkgreen}{\textbf{fSeekPdir:}} \textcolor{brown}{int32}
    \textcolor{blue}{\textbf{-}} \textcolor{darkgreen}{\textbf{fClassName:}} \textcolor{brown}{string}
    \textcolor{blue}{\textbf{-}} \textcolor{darkgreen}{\textbf{fName:}} \textcolor{brown}{string}
    \textcolor{blue}{\textbf{-}} \textcolor{darkgreen}{\textbf{fTitle:}} \textcolor{brown}{string}
\end{Verbatim}
\end{onlyenv}
\begin{onlyenv}<2>
\begin{Verbatim}[commandchars=\\\{\}]
\textcolor{darkgreen}{\textbf{TDirectory:}}
  \textcolor{darkgreen}{\textbf{members:}}
    \textcolor{blue}{\textbf{-}} \textcolor{darkgreen}{\textbf{version:}} \textcolor{brown}{int16}
    \textcolor{blue}{\textbf{-}} \textcolor{darkgreen}{\textbf{ctime:}} \textcolor{brown}{int32}
    \textcolor{blue}{\textbf{-}} \textcolor{darkgreen}{\textbf{mtime:}} \textcolor{brown}{int32}
    \textcolor{blue}{\textbf{-}} \textcolor{darkgreen}{\textbf{nbyteskeys:}} \textcolor{brown}{int32}
    \textcolor{blue}{\textbf{-}} \textcolor{darkgreen}{\textbf{nbytesname:}} \textcolor{brown}{int32}
    \textcolor{blue}{\textbf{-}} \textcolor{darkgreen}{\textbf{if:}}
        \textcolor{blue}{\textbf{-}} \textcolor{darkgreen}{\textbf{case:}} \textcolor{brown}{version <= 1000}
          \textcolor{darkgreen}{\textbf{then:}}
            \textcolor{blue}{\textbf{-}} \textcolor{darkgreen}{\textbf{seekdir:}} \textcolor{brown}{int32}
            \textcolor{blue}{\textbf{-}} \textcolor{darkgreen}{\textbf{seekparent:}} \textcolor{brown}{int32}
            \textcolor{blue}{\textbf{-}} \textcolor{darkgreen}{\textbf{seekkeys:}} \textcolor{brown}{int32}
        \textcolor{blue}{\textbf{-}} \textcolor{darkgreen}{\textbf{else:}}
            \textcolor{blue}{\textbf{-}} \textcolor{darkgreen}{\textbf{seekdir:}} \textcolor{brown}{int64}
            \textcolor{blue}{\textbf{-}} \textcolor{darkgreen}{\textbf{seekparent:}} \textcolor{brown}{int64}
            \textcolor{blue}{\textbf{-}} \textcolor{darkgreen}{\textbf{seekkeys:}} \textcolor{brown}{int64}
    \textcolor{blue}{\textbf{-}} \textcolor{darkgreen}{\textbf{keys:}}
        \textcolor{darkgreen}{\textbf{type:}} \textcolor{brown}{TKeys}
        \textcolor{darkgreen}{\textbf{at:}} \textcolor{brown}{seekkeys} \textcolor{mauve}{# seek to this value}
\end{Verbatim}
\end{onlyenv}
\begin{onlyenv}<3>
\begin{Verbatim}[commandchars=\\\{\}]
\textcolor{darkgreen}{\textbf{TKeys:}}
  \textcolor{darkgreen}{\textbf{doc:}} \textcolor{blue}{\textbf{|}}
    \textcolor{brown}{There is no ROOT class named}
    \textcolor{brown}{"TKeys," but it's useful to define}
    \textcolor{brown}{one here to represent a header}
    \textcolor{brown}{TKey followed by an arbitrary}
    \textcolor{brown}{number of TKeys.}
  \textcolor{darkgreen}{\textbf{members:}}
    \textcolor{blue}{\textbf{-}} \textcolor{darkgreen}{\textbf{header:}} \textcolor{brown}{TKey}
    \textcolor{blue}{\textbf{-}} \textcolor{darkgreen}{\textbf{nkeys:}}
        \textcolor{darkgreen}{\textbf{type:}} \textcolor{brown}{int32}
        \textcolor{darkgreen}{\textbf{at:}} \textcolor{brown}{$pos + header.keylen}
    \textcolor{blue}{\textbf{-}} \textcolor{darkgreen}{\textbf{keys:}} \textcolor{blue}{\textbf{\{}}\textcolor{darkgreen}{\textbf{array:}} \textcolor{brown}{TKey}, \textcolor{darkgreen}{\textbf{size:}} \textcolor{brown}{nkeys}\textcolor{blue}{\textbf{\}}}

\textcolor{darkgreen}{\textbf{TBuffer\_ReadVersion:}}
  \textcolor{darkgreen}{\textbf{doc:}} \textcolor{brown}{What TBuffer::ReadVersion does.}
  \textcolor{darkgreen}{\textbf{members:}}
    \textcolor{blue}{\textbf{-}} \textcolor{darkgreen}{\textbf{bytecount:}}
        \textcolor{mauve}{# needs to be transformed first}
        \textcolor{darkgreen}{\textbf{type:}} \textcolor{brown}{uint32}
        \textcolor{darkgreen}{\textbf{postprocess:}}  \textcolor{blue}{\textbf{|}}
            \textcolor{brown}{bytecount & ~uint32(0x40000000)}
    \textcolor{blue}{\textbf{-}} \textcolor{darkgreen}{\textbf{version:}} \textcolor{brown}{uint16}
\end{Verbatim}
\end{onlyenv}

\column{0.5\linewidth}
YAML is a declarative data language like JSON, but optimized for human input, often used for configuration files.

\vspace{0.25 cm}
With additional interpretation, we can use it as a DSL for describing ROOT layout.
\begin{itemize}
\item An {\it executable} specification!
\item Says nothing about eagerness vs.\ laziness of reading, leaving that to the implementation.
\item Resembles streamer info, apart from if-then branches.
\item Can add features as needed, but syntax is fixed.
\end{itemize}
\end{columns}
\end{frame}

\begin{frame}{Eagerness vs.\ laziness}
\vspace{0.5 cm}
Selective reading is one of the most important features of ROOT I/O.
\begin{itemize}
\item In C++, ROOT reads TKey and TBasket data in response to user requests.
\item In Python, I additionally want to avoid creating strings and other non-primitives before they're accessed (as Python ``properties'').
\end{itemize}

\vspace{0.25 cm}
\uncover<2->{These decisions are very implementation-dependent, and shouldn't be a part of a specification.}
\begin{itemize}
\item<2-> Although the YAML file describes the order of fields in the bytestream, they don't have to be read in that order.
\item<3-> As a demonstration, I implemented a {\it purely} lazy ROOT TH1F reader ({\it nothing} is read until explicitly referenced), configured by a YAML file:
\end{itemize}
\begin{center}
\uncover<3->{\href{https://github.com/jpivarski/rootspec}{\textcolor{blue}{https://github.com/jpivarski/rootspec}}}
\end{center}
\end{frame}

\begin{frame}{Streamer info}
Most ROOT classes are already specified to this degree, not in a document, but in the ROOT files themselves, as TStreamerInfo.

\vspace{0.5 cm}
Includes some very basic classes, like TTree, TList, TNamed, TObjArray\ldots

\vspace{0.5 cm}
No reason to duplicate this (and the version dependencies would be complicated to express in branches, anyway).
\end{frame}

\begin{frame}{How far can this kind of documentation go?}
\vspace{0.5 cm}

\textcolor{darkblue}{Brian's wish list:}
\begin{itemize}\setlength{\itemsep}{0.25 cm}
\item Container classes to ``bootstrap'' to the point where we can read streamers: TFile, TKey, TBasket, TDirectory, the streamer classes themselves\ldots

\item How STL classes are streamed (may rewrite STL documentation in our format).

\item ROOT's custom framing for compressed blocks (9 bytes before ZLIB, LZMA, and 17 before LZ4).

\item How streamers are generated from classes.

\item How classes are split into branches.

\item How cross-references are keyed by byte positions (relative to what origin).
\end{itemize}

\vspace{0.25 cm}
The last three could be hard to express declaratively\ldots
\end{frame}

\begin{frame}{Conclusions}
\vspace{1 cm}
A non-prescriptive specification of how ROOT I/O works would benefit data preservation, the process of changing ROOT I/O, and alternate implementations.

\vspace{0.5 cm}
\uncover<2->{Such a document can be an unofficial description (as Erik van Oosten did for Thrift).}

\vspace{0.5 cm}
\uncover<3->{It can also be executable, to verify that it's correct on test samples and to ensure that it doesn't get out-of-date.}

\vspace{0.5 cm}
\uncover<4->{I began a demonstration-level project documenting some core ROOT classes with YAML, which can configure readers anywhere on the eager-to-lazy spectrum (by implementing purely lazy; eager is much more straightforward).}

\begin{center}
\uncover<4->{\href{https://github.com/jpivarski/rootspec}{\textcolor{blue}{https://github.com/jpivarski/rootspec}}}
\end{center}
\end{frame}

\end{document}
